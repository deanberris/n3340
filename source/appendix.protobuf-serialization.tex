\section{Google Protocol Buffers Generic Serialization}
\label{appendix:protobuf-example}
An example of how a Google Protocol Buffer (protobuf) serialization mechanism
would work is shown below as a function that takes any protobuf object and
serializes it to an output stream.

\begin{verbatim}
// We want to ensure that the pointer we’re going to take does
// point to an object statically derived from proto::Message. For
// this we leverage the normal C++ static rules for inheritance
// and polymorphism.
namespace proto {
  bool serialize(rich_ptr<Message> m, std::ostream &os) {
    // In here we can then inspect the type associated with the pointer m
    // reflecting the runtime representation.
    type_descriptor_t const *type = type_descriptor(m),
                            *msg_type = type_descriptor<Message>(nullptr);

    // We then traverse the members of the protocol buffer and perform a generic
    // lookup of the data so we can encode properly into the output stream.
    std::string buffer;
    for (field_descriptor_t const *field : type->members) {
      if (field == nullptr)
        break;
      if (!m->get_member(field->name, &buffer))
        return false;
      os.write(s.data(), s.size());
    }
    return true;
  }
}
\end{verbatim}

For exposition and completeness, we provide an example implementation of
\verb+proto::Message+’s \verb+get_member+ function and enclosing scope using
rich pointers and introspection functionality:

\begin{verbatim}
namespace proto {
  class Message {
    char *buffer_;  // dynamically initialized by derived classes
    static std::map<std::string, type_descriptor_t const *> registry;

   protected:
    explicit Message(size_t derived_size)
    : buffer_(new (std::nothrow) char[derived_size]) {}

   public:
    virtual bool get_member(std::string const &name, std::string *buffer) {
      type_descriptor_t const *type =
          registry[type_descriptor(rich_cast<Message>(this))->name];
      bool ok = false;
      size_t offset = 0;
      field_descriptor_t const * field = nullptr;
      tie(ok, offset, field) = get_offset(name, type->members);
      if (!ok) return false;
      size_t size = field->type->size;
      buffer->assign(buffer_ + offset, buffer_ + size);
      return ok;
    }

    virtual ~Message() {
      delete [] buffer_;
    }
  };
}
\end{verbatim}


