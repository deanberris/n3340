\subsection{Rich Pointers}

The smart pointer idiom has been generally accepted in practice largely for
the utilities afforded to us by these smart pointers. The standard library now
contains three smart pointers: \verb+unique_ptr+, \verb+shared_ptr+, and
\verb+weak_ptr+. These smart pointers encapsulate generic patterns for dealing
with memory ownership and management and makes it transparent to users.

Following the smart pointer idiom this paper proposes a concept called rich
pointers which not only carry the memory location of a given object but also a
reference to an immutable representation of the type of this object (we call
these \emph{type descriptors}). An example usage of the \verb+rich_ptr+ is shown
here:

\begin{verbatim}
  struct foo {
      foo() {}
      ~foo() {}
  };

  rich_ptr<foo> p{new foo};
\end{verbatim}

For the most part a rich pointer behaves like a normal pointer. Dereferencing
will yield a reference to the object pointed to. A \verb+rich_ptr+ doesn't
imply ownership semantics and is only meant to associate the type descriptor
for a given normal pointer.

There are two primary operations supported on \verb+rich_ptr+ instances:
getting the type descriptor and performing a \verb+rich_cast+. The following
example illustrates the semantics of both these operations.

\begin{verbatim}
  rich_ptr<void> q = p;
  rich_ptr<foo> r = rich_cast<foo>(q);
  assert(type_descriptor(q) == type_descriptor(r));
\end{verbatim}

Because p and q point to the same object in memory, getting the type
descriptor of both rich pointers will yield the same type descriptor. In fact,
any object of type \verb+foo+ in memory when referred to via a rich pointer
will have the same type descriptor.