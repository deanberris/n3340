\subsection{Rich Pointers}

The smart pointer idiom has been generally accepted in practice largely for the
utilities afforded to us by these smart pointers. The standard library now
contains three smart pointers: unique\_ptr, shared\_ptr, and weak\_ptr. These
smart pointers encapsulate generic patterns for dealing with memory ownership
and management and makes it transparent to users.

Following the smart pointer idiom this paper proposes a concept called rich
pointers which not only carry the memory location of a given object but also a
reference to an immutable representation of the type of this object (we call
these type descriptors). The proposed syntax for a rich pointer follows the
general style of a pointer but instead of using the * symbol to denote a pointer
we use the percent ``\%'' symbol to do so.

\begin{verbatim}
  struct foo {
      foo() {}
      ~foo() {}
  };

  foo %p = new (std::rich) foo;
\end{verbatim}

For the most part a rich pointer behaves like a normal pointer. Dereferencing
will yield a reference to the object pointed to but you can no longer perform
normal pointer arithmetic. When a rich pointer is assigned to a normal pointer,
the memory location is transferred and the reference to the original type
descriptor is dropped (this is to ensure backward compatibility).

What rich pointers support that would be hard to support with normal pointers is
the notion of preserving type information even across casts even to
\verb+void %+. To illustrate more appropriately:

\begin{verbatim}
  void %q = p;
  foo %r = rich_cast<foo%>(q);
  assert(type_descriptor(q) == type_descriptor(r));
\end{verbatim}

Because p and q point to the same object in memory, getting the type descriptor
of both pointers will yield the same type descriptor. In fact, any object of
type foo in memory when referred to via a rich pointer will have the same type
descriptor.


