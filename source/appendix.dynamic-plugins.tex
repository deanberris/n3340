\section{Dynamic Plugins Example}
\label{appendix:plugins-example}

Here we show one approach to adding dynamically registered types as plug-ins to
a hypothetical server using rich pointers:

\begin{verbatim}
// We want to have a global registry of handler objects mapped to URLs.
static map<string, pair<rich_ptr<void>, function<void<string>>>> handlers;

// Then we define a way for a new module to register completely new
// objects at runtime:
bool register_handler(std::string const &url,
                      rich_ptr<void> handler,
                      bool replace) {
  // Here we can determine whether the object being registered as a
  // handler supports the required API (at runtime!)
  type_descriptor_t const *type = type_descriptor(handler);
  if (type == nullptr) return false;

  bool compliant = true;
  type_descriptor_t const *string_type =
      type_descriptor(rich_ptr<std::string>());

  function<void(string)> f;
  for (method_descriptor_t const *method : type->methods) {
    if (method == nullptr) break;
    compliant = compliant || (
        equals(method->type->name, “get”)
        && method->access_type == method_descriptor_t::PUBLIC
        && length(method->type->args) == 2
        && method->type->result_type == nullptr
        && method->type->args[1] == string_type);
    if (compliant && !f.get()) f = bind(method->type->callable, f, _1);
  }
  if (!compliant) return false;
  if (!replace && handlers.find(url) != handlers.end()) return false;
  handlers[url] = make_pair(handler, f);
  return true;
}

// When we’re ready to handle a URL, we do the following:
bool handle_url(std::string const &url, std::string const &request) {
  auto it = handlers.find(url);
  if (it != handlers.end()) {
    try {
      (*it)(request);
    } catch (...) {
      return false;
    }
    return true;
  }
  return false;
}
\end{verbatim}


