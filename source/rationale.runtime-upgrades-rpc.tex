\subsection{Runtime Upgrades and RPC}

In the realm of distributed systems programming the remote procedure call (RPC)
pattern is very popular and many implementations rely on a mix of static types
and dynamic structures to represent types. CORBA services rely on a very rigid
client and server interface described using a domain specific language to define
the interface by which clients can interact with the server. There are also a
number of solutions for web services like SOAP (using XML for encoding RPC
calls) and recently more popular REST interfaces (leveraging HTTP semantics and
usually passing data encoded in JavaScript Object Notation (JSON)).

CORBA and SOAP services are typically mechanically generated using compilers
that generate stubs later filled in by developers. These however suffer from the
API versioning problem largely because there’s no way for the server to
communicate with the clients the existence of new types and methods. Instead the
coupling between the client and the server APIs require upgrades to be
synchronized or in some cases have costly deprecation paths for client APIs that
require the service to keep supporting an old version and the new version at the
same time.

For web services written in C++ that use JSON as the data interchange format
between client code (typically HTML pages with JavaScript running on web
browsers) and server code, the only way to treat the data coming from the client
is to treat it as a stream of bytes that are parsed and maybe generate
statically typed objects to manipulate in the server. The other problem has to
do with streaming hand-rolled or dynamic types as JSON or another data
interchange format like XML.

See an hypothetical example of a JIT-compiling runtime-upgrading service that
relies on dynamically generated types that are inspected and again upgraded
using rich pointers in \autoref{appendix:jit-rpc-example}.
