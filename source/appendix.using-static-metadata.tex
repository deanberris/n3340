\section{Appendix: Using Static Metadata}
\label{appendix:example-2}

In this section we show how we can use the static metadata. We start with a minimal example similar to \autoref{appendix:example-1} where we print the name of the type.

\begin{verbatim}
  #include <iostream>
  #include <metadata>  // proposed header.
  
  struct foo {};
  
  int main(int argc, char* argv[]) {
    using namespace std;
    // The following should print "::foo"
    cout << name<metadata<foo>>::value << endl;
    return 0;
  }
\end{verbatim}

Because we can make inquiries on the types enabled by \verb+metadata+ we can use this in SFINAE as well as in \verb+static\_assert+ situations:

\begin{verbatim}
  template <class T>
  constexpr void range_concept_check(T) {
    static_assert(name<metadata<T>>::value != nullptr, "We need static reflection!");
    
    struct begin { static constexpr const char* value = "begin"; };
    struct end { static constexpr const char* value = "end"; };
    
    // We look for a function named begin and end that return the same types.
    // In this part we hand-wave away some metafunctions.
    static_assert(
      is_same<
        result_type<
          find<
            functions<metadata<T>>,
            equals<
              name<_>::value,
              begin::value
            >
          >
        >,
        result_type<
          find<
            functions<metadata<T>>,
            equals<
              name<_>::value,
              end::value
            >
          >
        >
      >::value, "To be a range, begin and end should return the same type.");
  }
\end{verbatim}