\section{Type Registration and Invalidation}
\label{appendix:type-registration-invalidation}

There are two functions that are being proposed as standard extensions in the
\verb+<runtime>+ header: \verb+register_type+ and \verb+invalidate_type+.

\verb+register_type+ is meant to be called at module initialization time
depending on how dynamic/shared or static modules are initialized.  The result
is a boolean indicating success and a pointer to the new immutable type
descriptor. This will only return true if the type T is valid and that it has
not been previously registered.

\begin{verbatim}
  // Example usage:
  //     bool ok;
  //     type_descriptor_t const *p;
  //     tie(ok, p) = register_type(new(std::rich) foo);
  //     assert(ok);
  template <class T>
  tuple<bool, type_descriptor_t const *>
  register_type(T%);
\end{verbatim}

\verb+invalidate_type+ is meant to be called at module initialization time
depending on how dynamic/shared or static modules are initialized.
The result contains a boolean indicating success and pointers to the
invalidated type descriptor and new type descriptor. It is important
that this is called at module initialization time when the current
runtime context and module-specific runtime contexts are available.
Calls to \verb+new(std::rich)+ within module scope will always use the
module-specific runtime context. During module initialization, calls
to \verb+type_descriptor(...)+ will always refer to the current runtime
context. If a type is not already registered through
\verb+register_type(...)+ then calling \verb+invalidate_type+ will return the tuple
(\verb+false+, \verb+nullptr+, descriptor to new type).

\begin{verbatim}
  template <class T>
  tuple<bool, type_descriptor_t const *, type_descriptor_t const *>
  invalidate_type(type_descriptor_t const *, T%,
                  function<bool(type_descriptor_t const *, T%&)>);
\end{verbatim}

The following listings show example example usage of the \verb+invalidate_type+
function.

\begin{verbatim}
  bool ok;
  type_descriptor_t const *old_type, *new_type;
  foo %old = nullptr;
  tie(ok, old_type, new_type) =
      invalidate_type(type_descriptor(old), new(std::rich) foo);
  assert(new_type != old_type && ok);
\end{verbatim}

Another case is for ``deregistering'' a type, by simply providing nullptr to the
second argument to \verb+invalidate_type+:

\begin{verbatim}
  tie(ok, old_type, new_type) =
      invalidate_type(type_descriptor(old), nullptr);
  assert(new_type == nullptr && type_invalidated(old) && !ok);
\end{verbatim}

The above is suggested for module cleanup time when types only meant for module
scope should be deregistered and all rich pointers for these types should be
invalidated.

The third argument to \verb+invalidate_type+ is the invalidation handler
function. This function is called when a rich pointer of an invalidated type is
dereferenced. It is passed the old type descriptor and a reference to the
invalidated rich pointer.  The invalidation handler should return true if it
should allow the application to continue execution and false to force a call to
\verb+std::abort+. The default handler throws an \verb+invalidated_ptr+
exception which contains the invalidated pointer and a pointer to the old type
descriptor.

A convenience function for explicitly checking whether the type of a given
pointer has been invalidated.

\begin{verbatim}
  bool type_invalidated(void%);
\end{verbatim}

There are only ever at most two versions of a type descriptor for any given
registered type.


