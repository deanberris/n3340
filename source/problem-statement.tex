\section{Problem Statement}

There are some application areas today where rich runtime information about
objects is crucial.  These application areas include:

\begin{itemize}

  \item \textbf{Dynamic Adaptation.} Highly available server applications that
  have to stay up as a long running process and enable for dynamic adaptation
  to changing requirements. Upgrades to these applications currently need to
  be closely coordinated and tightly controlled, usually requiring that the
  service be actually brought down when an upgrade is required. There are
  solutions that exist which involve dynamically loaded libraries but changing
  the design of already packaged types in the system usually require downtime
  to rebuild and relink the binary.

  \item \textbf{Embedded Dynamic Languages.} Applications that embed dynamic
  programming languages typically settle for either a static interface to
  which the embedded dynamic language runtime, or a pure data interface
  implementing a (static) protocol between the embedded environment and the
  host application. Usually the dynamic types that can be generated in these
  embedded virtual machines are only usable in the context of these virtual
  machines limiting the interactions by which these embedded types interact
  with the host application.

  \item \textbf{Distributed Computing.} Distributed computing systems written
  in C++ are largely tied to static interfaces and types because of the
  limitation of the programming language. We currently do not have a standard
  programmatic way of dynamically generating types and referring to objects of
  these types and streaming objects of these types from one system to another.
  Current state of the art relies on code generators, domain specific
  languages, and even communication frameworks to achieve remote procedure
  calling and sharing state across elements in a distributed system.

  \item \textbf{Data Structure Refinement.} Applications use machine learning
  and dynamic modeling of environments through continuous refinement of data
  structures in memory rely on the capability to treat code as data, or at
  least be able to inspect the state and relationship between types in a
  hierarchy of types. The current limitations of the language force the
  applications that deal with constantly changing structures and types is to
  model them in runtime as merely data, losing much of the power of the
  programming language’s runtime facility for efficiently modeling types and
  objects in the process.

  \item \textbf{Dynamic Characterization.} Graphical user interfaces typically
  rely on being able to inspect in­-memory structures to represent graphical
  elements to be rendered on screen. Almost all the sufficiently advanced
  graphical user interface toolkits now use a statically­defined and very
  rigid type system and implement a runtime meta­type system because the
  programming language currently does not have a facility of doing proper rich
  runtime introspection of types.

\end{itemize}

There are several problems that this paper aims to address. Here are some of
these problems:

\begin{itemize}

  \item \emph{How do we find out what the type of an object is at a given
  memory location?} Type erasure allows us to expose a generic API that works
  well across module boundaries, but being able to preserve type information
  across these module boundaries without having to rebuild and relink binaries
  is also as powerful.

  \item \emph{How do we print the structure of a dynamic type at runtime?}
  Currently there are no ways to do this dynamically without resorting to
  manual, static, and expensive checks on types that are members of a
  statically-known hierarchy of types. There exists no standard means of
  knowing the type of a given object in memory at runtime with the current
  features of the language, especially when referred to using either a
  \verb+void *+ or a base type pointer.

  \item \emph{How do we determine the relationship between any two given types
  at runtime?} The current way of doing this requires manual checks for
  whether one type can be dynamically cast to another, and inferring from the
  result what the potential relationship between the types are. There is
  currently no way of determining what types a given type is derived from,
  what type of inheritance (public, private, protected, virtual, etc.), and
  whether a given type is abstract or final, etc.

  \item \emph{If we were able to create new types at runtime, how do we
  describe these types and inspect them?} For applications that rely on live
  updates for high availability and remote procedure calling systems, being
  able to reconstitute a type dynamically based on data obtained externally
  via I/O is crucial. Currently the only way to allow this is to implement a
  runtime type system by hand and perform all type inspections by hand, unable
  to leverage the rich type system that C++ provides. This is also important
  in applications where an embedded just-in-time (JIT) compiler can create new
  structures programmatically as well as allowing types generated in embedded
  runtime environments to be exported to the host application.

\end{itemize}
