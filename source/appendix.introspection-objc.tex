\subsection{Objective-C}

Objective--C\cite{objc} is a SmallTalk--derived object--oriented layer added
to standard C. Originally developed by Brad Cox and the StepStone
Corporation\cite{dekorte}, it was licensed by NeXT and then reimplemented by
Dennis Glatting, Richard Stallman, and Kresten Krab Thorup. The last
implementation has been the official GNU runtime since 1993\cite{dekorte}.
Currently, it's used as the vendor--preferred API for new application
development on Mac OS X. The primary compiler is the Free Software Foundation
GNU Compiler Collection (GCC).

Objective--C provides a dynamic runtime environment running on the native
processor; no virtual machine is used. Although there is no garbage
collection, reference counting is built into the libraries, and the burden on
the programmer is minimal to interoperate with it. Furthermore, the runtime
provides interesting functionality, such as the ability to extend a class at
compile, link, or run time. Finally, the runtime interacts with the client
code to enable additional capabilities, discussed later.

Objective--C has been used for years for next--generation systems like
NeXTStep and WebObjects, and has always enjoyed good tool support. Apple
provides almost all support for Objective--C today. The APIs have changed
names, from an \texttt{NX} prefix to \texttt{NS}. Today, both
GNUStep\cite{gnustep} and Apple's implementations use the new prefix.

\subsubsection{Compilation}	

Objective--C compiles like C or C++: a set of source files (with the
\texttt{.m} suffix) are compiled to object files. The source and object files
have no required naming relation to the code within. The compiler stores
additional class metadata describing the methods, members, and layout of the
aggregates defined.

The code first goes through a preprocessor like C. The Objective--C
preprocessor is nearly identical to C's, except for an additional
\texttt{\#import} directive, which works like \texttt{\#include} with
built--in multiple include guards.

Executables generated from Objective--C are traditional in nature;
\texttt{ELF}, \texttt{COFF}, and \texttt{Mach-O} are supported in modern
toolchains. NeXTStep and Mac OS X package executables into directories called
bundles, which are treated as a single file within the user interface.

Like executables, libraries are also generated like any other on the platform:
\texttt{DLL}, \texttt{.so}, or \texttt{.dylib}. NeXTStep and Mac OS X package
libraries into directories called frameworks, again treated as single files
within the user interface. Application bundles and frameworks can contain each
other as needed.

While Objective--C provides no direct application management infrastructure,
its cohorts NeXTStep and Mac OS X, through their bundling and framework
mechanisms, allow drag \& drop application installation and removal.

\subsubsection{Syntax}

Objective--C's syntax is essentially C with a modified version of SmallTalk
put on top. Like C++, a header file declares the type's messages and members.
The keyword \texttt{@interface} specifies the name of the type, its supertype,
and any protocols it implements.

The primitive types are 100\% C. Instances of objects aren't allowed on the
stack, only pointers. All of C's declaration syntax is supported, and pointers
to objects are declared as they would be in C++. Objective--C also supports
typeless object types called \texttt{id}. It's actually a typedef to
\texttt{void*}, but is allowable everywhere an object pointer would otherwise
be required.

The syntax for loops, conditionals, and free functions is identical to C. This
allows conditionals to check pointer values directly, without a redundant
\texttt{==0} like Java. Even though free function definitions and calls to
them are identical to C, message sending and method definition is very
different. Those are based on SmallTalk.

\subsubsection{Type System}

As mentioned earlier, all of C's primitives are supported. Structures are also
supported. Wrappers for them exist, although fewer in number than Java.
\texttt{NSNumber} covers all numeric types. \texttt{NSString} and
\texttt{NSMutableString} cover strings. Through them, one can put primitive
types inside containers and use them as regular objects.

Beyond that, Objective--C has two interrelated constructs, the
\texttt{@interface} and the \texttt{@implementation}. The former
declares the members and the list of methods visible to clients. The latter
contains all the method implementations. Note that additional methods can be
defined in the \texttt{@implementation}, which will be added to the method
list as any other. These additional methods can be called by any client due to
the dynamic messaging system.

C's arrays are immediately available. Furthermore, \texttt{NSArray} and
\texttt{NSMutableArray} provide object wrappers around fixed arrays.
Objective--C's containers all provide useful abstractions upon all their
contents, and these two types enable them for arrays. Arrays of arrays are
allowed in C's traditional arrays and in Objective--C. \texttt{NSArray} and
\texttt{NSMutableArray} allow them as well by recursive embedding.

Objective--C's can operate with weak or semi--weak type checking. The original
libraries and runtime use the type \texttt{id} to refer to any object. Any
message could be sent to it, and any type is convertible to \texttt{id}.
Internally, \texttt{id} is a \texttt{typedef} to \texttt{void*}. However,
users complained about the lack of static error checking, so some basic
capabilities were added.

A pointer can be declared to any \texttt{@interface}, and messages sent to
that pointer are checked against the \texttt{@interface}'s declared API (not
the set of messages the \texttt{@implementation} defines). Any messages sent
to the object, but not declared by its API, are flagged as warnings by the
compiler. Because the object could easily respond to many more messages than
it declares, a compilation--stopping error would be inappropriate.

For primitive types, the type checking is exactly the same as standard C.
Traditional \texttt{(type)} casts easily remove any hindrances provided by the
type system.

Objective--C's object model is modeled after SmallTalk. Only single
inheritance is supported, but with two other important features: categories
and protocols.

Categories were originally designed to let developers split up the
\texttt{@implementation}s of types across several source files. They
allow additional methods to be added to an existing type. Furthermore, a
category's implementation of a method will \emph{override} the original. Note
that this isn't the same as overriding a method through inheritance; the
category's methods are part of the original type. The overridden method is
never used. If two or more categories implement the same method, then the
selection of the method implementation that runs depends upon the load order
of the categories' respective object definitions at run--time.

Protocols work essentially like \texttt{interface}s in Java: a set of methods
that are declared but undefined. Classes declare their conformance to one or
more protocols and implement their methods. Due to the fairly untyped nature
of Objective--C, the actual declaration of conformance is mostly for
documentation reasons. Just as often as protocols are used, classes will
declare \emph{informal protocols}. Informal protocols exist only in
documentation as a set of methods a class may choose to implement.
Objective--C's runtime allows a class's implemented method list to be checked,
so a client can see if it implements a method from an informal protocol before
sending it the appropriate message.

At the heart of Objective--C's runtime is the \texttt{Class}. The
\texttt{Class} contains the full description of a class: its superclass, name,
version, size, instance variables, method list, and conformed protocols. The
\texttt{Class} is a \texttt{typedef} to \texttt{struct objc\_class*}, a C
structure. While the definition of \texttt{objc\_class} is available through
the header files, several C and Objective--C functions and methods are
available to the programmer for abstracted access.

The \texttt{Class} contains the list of methods implemented by the type. All
method names are hashed to an opaque \texttt{struct objc\_selector*},
generally just called a \emph{selector}.

Every instance of the same name, no matter which interface or implementation
it exists in, is hashed to the same value. This way, message names have no
binding to any particular class. Using this property, one can send any message
to any type.

Upon compilation, the compiler links to the runtime in two ways. First, the
\texttt{Class} structures are all defined and filled. Second, every message
send has the method name converted to a selector, and then passed to the
runtime function \texttt{objc\_msgSend}.

\texttt{objc\_msgSend} searches the recipient's method list for the selector,
and dereferences the listed pointer to the proper code. The method list is
actually a hash table for speed. If the object doesn't implement the method,
all of its ancestors are searched. If none of them implement it, then it sends
a ``second chance" invocation, by sending \texttt{forwardInvocation:} to the
original recipient.

\texttt{forwardInvocation:} allows an object to handle any message it wants.
The name implies the most common use for this feature: to let an object act as
a proxy for another. An object can implement \texttt{forwardInvocation:} as a
quick check upon its proxied object to see if it responds to the parameter,
and if so, sends the message onwards. \texttt{forwardInvocation:} has a
default implementation in the root class \texttt{NSObject} that throws an
exception and logs the failed message call.

Objective--C doesn''t provide any type of parameter overloading at all. In
fact, parameter checking isn't a strong feature at all in Objective--C. Only
when the optional static type checking features are used can parameter checks
be done at all.

A message is dispatched upon its selector, which is based only on the message
name. When static type checking isn't used, the return type is assumed by the
runtime to be of type \texttt{id}. Any client sending a message to a type that
returns anything else has to cast the value.

As mentioned earlier, a form of overloading occurs with categories that define
methods with the same name as one in the original type. This is a dangerous
practice, as another category could do the same and ambiguate the selection
process.

Objective--C only allows access control upon members. Although one can ``hide"
methods by defining them in the \texttt{@implementation} without declaration
in the \texttt{@interface}, instances will still respond to messages of that
name from any sender.

Members can still be \texttt{@public}, \texttt{@protected}, or
\texttt{@private}. The syntax for accessing them is identical to accessing a
structure member via a pointer. For further control, the traditional C methods
of defining pointers to undefined (such as \texttt{objc\_selector}) or
unlisted types (\texttt{void*}) are still useful.

As mentioned earlier, Objective--C allows a type to inherit from one other.
All the member variables and methods are inherited. \texttt{@private} members
are part of the subclass's definition but are inaccessible. All methods are
inherited and accessible. Any method can be overridden, as there is no concept
of a \texttt{final} or non-virtual method in Objective--C.

While upcasting is directly allowed, it's not very useful. C allows arbitrary
casting and Objective--C doesn't really need a type's name to send it a
message. Except for cases of voluntary static type checking, inheritance in
Objective--C is essentially useful as implementation inheritance only.

Objective--C provides the most flexible polymorphism of any language described
here. It has a binding mechanism much more dynamic than most OO languages.
Furthermore, it allows the \emph{client} much more control over an object's
behavior than most other languages: categories and dynamic message proxies let
the client modify the interface and behavior of the type. Only
\texttt{@private} data members aren't accessible to client code via
subclassing or categorization.

Exceptions only appeared in their current form in Apple's compiler in Mac OS X
10.3. An earlier form existed that was based on macros, but it won't be
described here. The current syntax is very similar to C++: a \texttt{@try}
block encloses code that may \texttt{@throw}, followed by one or more
\texttt{@catch} blocks, each with a typed parameter. Like Java, an optional
\texttt{@finally} block can follow, with code that will always be run.

Only subclasses of \texttt{NSException} may be thrown. The \texttt{@throw}
construct begins a scan and unwind process on the stack for a \texttt{@catch}
block that will handle the exception type or one of its ancestors. Any
\texttt{@finally} blocks are run along the way.

Objective---C doesn''t have a compile--time generics system. Its dynamic
messaging system helps mitigate the loss. For example, the library containers
all support the method \texttt{makeObjectsPerformSelector:}, which sends a
message with the parameter selector to every contained element. Furthermore,
the weak and optional static typing mechanism reduces the value of a generics
facility at all.

\subsubsection{Runtime}

As covered earlier, the runtime plays a large part of the Objective--C system.
At startup, all the loaded classes' \texttt{Class} descriptions are
initialized. From there, the standard C entry point \texttt{main} is run.

Memory is best described as ``semiautomatic." Its inherently
reference--counted, with the appropriate support infrastructure directly
within \texttt{NSObject}. Two methods, \texttt{retain} and \texttt{release},
increase and decrease a reference count in the recipient object. When that
count reaches zero, the object calls \texttt{dealloc} on itself, and releases
any resources it holds. \texttt{dealloc} will then call \texttt{release} on
any member objects it has, causing the proper cascading effect.

For additional functionality, Objective--C supports an
\texttt{aut\-or\-el\-ease} message. \texttt{autorelease} will add the receiver
to the current autorelease pool, an instance of \texttt{NSAutoreleasePool}, a
container. \texttt{NSAutoreleasePool} takes ownership of the object. When the
pool is itself \texttt{release}ed, it will call \texttt{release} on all of its
contents.

\texttt{NSAutoreleasePool}s can be nested within each other, and are
automatically supported by the GUI library in Objective--C, \texttt{AppKit}
(now called Cocoa\cite{cocoa}). The most common use is within the event loop:
an autorelease pool is created when an event enters the application, and
\texttt{release}ed when the event loop reenters it waiting state. In the
meantime, any \texttt{autorelease}d object will have a convenient lifetime
that lasts just long enough to be used for handling the event. Any object that
needs to live longer can be \texttt{retain}ed by any other object, which will
hold the only reference count to it when the event loop reenters the waiting
state.

While not as large as Java''s, Objective--C''s library is certainly quite
powerful. More importantly, it's much more flexible thanks to good leverage of
the runtime's dynamism. For example, the serialization mechanism directly
allows connections between objects to be serialized easily, allowing a GUI
builder to directly connect data objects to GUI components without
necessitating the generation of code or other glue. Furthermore, serialization
works for nontree structures, thanks the the runtime's ability to directly
analyze the objects.

Of particular interest in the library is the single--threaded nature of it
all. The entire system was designed to run within a single thread, with a
messaging bridge to connect threads together. The messaging bridge works
exactly the same as a full interprocess communication system, in fact
identically to its distributed objects mechanism. That will be described
below.

Most of the runtime's abilities come from two places: the messaging system and
functionality built into \texttt{NSObject}. \texttt{NSObject} provides the
following APIs:

\begin{enumerate}

 \item \texttt{conformsToProtocol:} --- If the object exports a specific
protocol's API.

 \item \texttt{respondsToSelector:} --- If the object responds to a specific
method.

 \item \texttt{isKindOfClasss:} --- If the object has a specific class as its
type or in its type inheritance hierarchy.

 \item \texttt{isMemberOfClass:} --- If the object is of a specific type.

\end{enumerate}

These methods constitute around 90\% of what's needed by most applications.  The remainder go and directly analyze the \texttt{objc\_class} structure for more information.  For the most part, the dynamic messaging infrastructure and weak typing reduce the need for direct introspection in the system.

\subsubsection{Case Study: Distributed Objects}

There need not be any ``what if" scenarios here; Objective--C's basic runtime
libraries provide distributed object (DO) functionality. In fact, this
functionality's used for more than interprocess communication; Objective--C
uses its DO mechanism for inter-thread communications; avoiding the
traditional complexities of synchronization.

The distributed object system works quite simply: a generic proxy class for
the transport mechanism masquerades as the type it's proxying. The proxy tells
its clients that it responds to all the selectors of its destination. Any
messages sent to it are serialized and sent through the transport to the
recipient. The system is so simple because of both the flexible messaging
system and the built--in serialization system.

The serialization of messages is significant; such a serialized form could be
replicated, replayed, or even analyzed and restructured. For a DO system,
another question arises, when should a parameter be serialized and when should
it be proxied? If the parameter is a simple, flat data structure, it can and
should be serialized. However, if it's a complex object with relationships to
other objects, it should be proxied. Serializing and restoring such an object
will certainly be complex, possibly lossy. Such a determination also can't be
made by the compiler; these are design--level decisions.

For that reason, additional keywords exist in Objective--C to tag parameters
that should be proxied. The terms \texttt{byref} and \texttt{bycopy} indicate
which parameters are sent by reference and copy, respectively

\subsubsection{Case Study: Serialization}

Serialization is built into Objective--C as well; it's a prerequisite for
marshaling a message. There are two parts to serialization: wrappers for each
primitive type and a protocol for serialization. The latter, \texttt{NSCoding}
lets serializing objects interoperate and cooperate for more complex types.
The wrappers implement the protocol and take most of the grunt work out of
serialization.

As Objective--C doesn't allow the iteration of an object's members, there is
no support for automating the serialization process of a complex type; some
code has to be written. In practice, the process is often much simpler than it
sounds. Objective--C's runtime library is full of high--quality containers
that will serialize themselves and all their contents. Often, it's
advantageous to use one or two containers instead of many members; letting
them do most of the serialization work.

%
%
%
