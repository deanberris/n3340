
\section{Introspection in Programming Languages}
\label{appendix:introspection}

Introspection is a major capability missing from C++. With it, software
components have the ability to simultaneously interact more intimately and
more generically.

For example, the JavaBeans \cite{javabeans} architecture uses a combination of
introspection and naming conventions to enable the automatic lookup,
discovery, instantiation, and initialization of Java classes from a data
definition that�s only available at class time. One common use is for an
application to read in a row of a table in a database and call setY on an
entity class (called a Bean in Java terminology) for every column Y in the
row. The application�s code can use metadata from the database to get all the
columns available, and introspect the bean to find and call the appropriate
initialization method.

Java, Smalltalk, and Objective-C provide introspection via runtime data
structures. Such data structures are generated for each class, method, and
member. For large systems, the storage overhead can become significant. Also,
the introspection requires runtime access to work; adding overhead to the
system�s runtime.

OpenC++\cite{chiba} provides a mechanism for compiler plug--ins to extend the
language itself.

\subsection{SmallTalk}
%+   Intro                       
%  -   Influence                     
%  -   History                     
%  -   Intent                      
%  -   Features                      
	SmallTalk derives from Simula and Lisp.  Originally developed at the Xerox Palo Alto Research Center in 1972, it did not leave the lab until 8 years later as SmallTalk-80. %  Designed for children aged 10 and above\cite{dev.smalltalk.org/zoku}, 
	It's based upon the idea that everything is an object, and that all objects communicate via messages.

	Simple constructs, like code blocks, integers, strings, conditionals, and looping, are library objects.  Through extensive reuse and inheritance, it's easy to leverage the extensive and highly--integrated existing library objects.  

\subsubsection{Compilation}
	SmallTalk comes in more than just a compiler, it's an entire environment.  When source text is compiled, the resulting objects are inserted into the environment, ready for immediate use.  For specifics, we'll discuss the specifics of one particular SmallTalk environment, Squeak \cite{squeak}.

	The environment is saved and restored from an image file, that's bit--for--bit identical across platforms.  An application in Squeak is a set of object instances in the environment.  More can be instantiated as needed.

	More than one image can be loaded at a time and objects can be moved and copied between them.%\footnote{Completely guessing here}.  
Applications are packaged as images that are copied into new environments as needed.  Once copied, they are part of the new environment, and as such, can be invoked as if they were a built--in feature of that environment.
%+   Compilation                   
%  -   Compilation process               
%  -   Preprocessing stages                
%  -   input & output                    
%  -   archives & libraries                  
%  -   application management                

\subsubsection{Syntax}
	SmallTalk's syntax is very simple: a way of specifying messages sent to objects.  Such message send actions can be grouped together into blocks, which themselves can respond to messages.  The syntax specifies three ways to send messages to objects: unary, binary, and keyword.

	Unary messages have a simple format: \emph{recipient} \texttt{message}.  The recipient indicates which object to be sent the message, and the latter is the message to be sent.  Binary messages are almost as simple: \emph{recipient} \emph{message} \emph{parameter}.  The first two are the same as before, but the last value, the parameter, is given with the message to the recipient.

	Keyword messages are similar to method invocations in Java or C++.  The syntax is similar to the others: \emph{recipient} [\emph{namesegment:} \emph{parameter}]+.  The message would have a name like \texttt{setMax:min:}, with parameters following each of the colons.

	To force precedence, parenthesis are used.  These are necessary to send messages to objects returned from other messages.  For example: \texttt{array at:1 set:5.} would send the message \texttt{at:set:} to \texttt{array}, while \texttt{(array at:1) set:5.} would send the message \texttt{at:} to \texttt{array}, and the object returned would then receive the message \texttt{set:}.

	Variables have no static type in SmallTalk.  Variables are all references to objects, and all objects are typed.  Without static type, variable declarations are simply statements of their name in the proper contexts, such as member, variable, and parameter declarations.

	Conditionals are very simple: relational operators return objects of type \texttt{Boolean}, which has the method \texttt{ifTrue:ifFalse:}.  It takes two parameters, code blocks that run, depending on the value of the boolean value.  Loops work similarly: a block of code is an object that has a method called \texttt{whileTrue:}, which takes a block as a parameter.  That parameter block is run repeatedly as long as the block's code results in \texttt{true}.

	%One can create toplevel procedures in SmallTalk by creating \texttt{BlockClosure} objects (blocks of code), and assigning them to names.  Then those variables can be sent the \texttt{value:} message later to evaluate them.
	
%+   Syntax                      
%  -   Overview                      
%  +   Procedural syntax                 
%    -   Declarations                    
%    -   Conditionals                    
%    -   Loops                     
%    -   Procedure calls                 

\subsubsection{Type System}
	All data are within objects.  Arithmetic types are thus also objects, which implement methods for mathematical operations.  Precedence for operations is very simple: left to right, except for parenthesis. 

	There is no ``native" array concept in SmallTalk\cite{567553}; instead, a \texttt{Array} object exists, that uses methods \texttt{at:} and \texttt{at:put:} to read and write its values.  Two dimensional arrays are implemented with class \texttt{TwoDArray}, and higher dimensions are only possible with embedding \texttt{Array}s within each other.

	As SmallTalk doesn't give type to its variables or method parameters, no static type checking can be performed.  Only when an object receives a message it doesn't implement or a method throws an exception is an error flagged.  In fact, the former will actually just throw an exception of type \texttt{doesNotUnderstand} .

	
%+   Type System                   
%  -   Primitive Type Handling (integer, Integer, etc)       
%  -   Arrays                      
%  -   Type Safety/Checking                

\subsubsection{Object Model}

	Objects have a type, members, and methods.  The type is another object by itself, called the \emph{Class Object}\footnote{\cite{74921} extended this concept further into \emph{explicit metaclasses}\cite{74911}, resulting in the Classtalk platform}, referred to by a global variable with the type's name.  To subclass a type, send that type's class object a \texttt{subclass:\-instance\-Variable\-Names:\-class\-Variable\-Names:\-pool\-Dictionaries:\-category:} message, which creates a new class object with the added instance, class, and pool variables.

	To construct an instance, call \texttt{new} on the class object.  The return value is a new instance, which can then be used as necessary.  A garbage collector will automatically destroy any unused instances lying around.

	Operators are methods themselves, with non--alphanumeric names.  The expression $1 + 2$ leads to the instance of \texttt{Integer} for $1$ receiving the message \texttt{+} with a parameter of the instance of \texttt{Integer} for $2$.

	There isn't a heavily--enforced access control mechanism like in C++ or Java.  Instead, methods are placed into named categories, and some categories have the prefix \texttt{private}, which the runtime and compiler do not treat specially.  The prefix \texttt{private} simply means that the interfaces and implementations may change in a later version, so it's best not to use them.

	Inheritance and polymorphism exist in SmallTalk.  Single inheritance only\cite{802792}, without the concept of formal interfaces as in Java\footnote{Without typed variables, such a mechanism would not be useful anyways}.  Polymorphism is ubiquitous; to override a method, simply define it in the subclass.  There is no special keyword to mark a method as overridable like \texttt{virtual} in C++, nor is there a method to prevent override like \texttt{final} in Java.  Without access control or specially marked polymorphic members in SmallTalk, the language presents no obstacle to arbitrary changes of a type through subclassing.

	
%	SmallTalk does have some exception support, but it's almost .  XXX Research this and fill in, there's lots of differing data on this.  Perhaps SmallTalk-80's support was weak and has been augmented by vendors?

	SmallTalk has no support for generics.  As variables are untyped, it's understandable to avoid the issue.
	
	
%  +   Object Model                    
%    -   Messaging                   
%    +   Overloading                   
%      -   Operators                     
%    -   Access Control                    
%    -   Inheritance                   
%    -   Polymorphism                    
%  -   Exceptions                    
%  -   Generics             

\subsubsection{Runtime}

	The SmallTalk system is a full environment that contains all objects and interacts with the user.  Programs in SmallTalk exist within \emph{images}, virtual machine memory segments that can be added together.  Objects are fully persisted within the images, and they live until garbage collected.

	The SmallTalk libraries are extensive and easy to use.  The language has a built--in documentation system, where every object and method can have a string associated with them, which can then be browsed.  A programmer can use an object browser to peruse the extensive library of collections, utilities, graphics, and network classes.

	SmallTalk does provide introspective abilities through \texttt{ClassDescription}.  It's possible to retrieve all method names, categories, and members through its APIs.  As there is no access control, all data is available to every client.

%+   Runtime                     
%  -   Startup                     
%  -   Memory Management               
%  -   Libraries                     
%  -   Introspection                   

\subsection{Objective-C}

Objective--C\cite{objc} is a SmallTalk--derived object--oriented layer added
to standard C. Originally developed by Brad Cox and the StepStone
Corporation\cite{dekorte}, it was licensed by NeXT and then reimplemented by
Dennis Glatting, Richard Stallman, and Kresten Krab Thorup. The last
implementation has been the official GNU runtime since 1993\cite{dekorte}.
Currently, it's used as the vendor--preferred API for new application
development on Mac OS X. The primary compiler is the Free Software Foundation
GNU Compiler Collection (GCC).

Objective--C provides a dynamic runtime environment running on the native
processor; no virtual machine is used. Although there is no garbage
collection, reference counting is built into the libraries, and the burden on
the programmer is minimal to interoperate with it. Furthermore, the runtime
provides interesting functionality, such as the ability to extend a class at
compile, link, or run time. Finally, the runtime interacts with the client
code to enable additional capabilities, discussed later.

Objective--C has been used for years for next--generation systems like
NeXTStep and WebObjects, and has always enjoyed good tool support. Apple
provides almost all support for Objective--C today. The APIs have changed
names, from an \texttt{NX} prefix to \texttt{NS}. Today, both
GNUStep\cite{gnustep} and Apple's implementations use the new prefix.

\subsubsection{Compilation}	

Objective--C compiles like C or C++: a set of source files (with the
\texttt{.m} suffix) are compiled to object files. The source and object files
have no required naming relation to the code within. The compiler stores
additional class metadata describing the methods, members, and layout of the
aggregates defined.

The code first goes through a preprocessor like C. The Objective--C
preprocessor is nearly identical to C's, except for an additional
\texttt{\#import} directive, which works like \texttt{\#include} with
built--in multiple include guards.

Executables generated from Objective--C are traditional in nature;
\texttt{ELF}, \texttt{COFF}, and \texttt{Mach-O} are supported in modern
toolchains. NeXTStep and Mac OS X package executables into directories called
bundles, which are treated as a single file within the user interface.

Like executables, libraries are also generated like any other on the platform:
\texttt{DLL}, \texttt{.so}, or \texttt{.dylib}. NeXTStep and Mac OS X package
libraries into directories called frameworks, again treated as single files
within the user interface. Application bundles and frameworks can contain each
other as needed.

While Objective--C provides no direct application management infrastructure,
its cohorts NeXTStep and Mac OS X, through their bundling and framework
mechanisms, allow drag \& drop application installation and removal.

\subsubsection{Syntax}

Objective--C's syntax is essentially C with a modified version of SmallTalk
put on top. Like C++, a header file declares the type's messages and members.
The keyword \texttt{@interface} specifies the name of the type, its supertype,
and any protocols it implements.

The primitive types are 100\% C. Instances of objects aren't allowed on the
stack, only pointers. All of C's declaration syntax is supported, and pointers
to objects are declared as they would be in C++. Objective--C also supports
typeless object types called \texttt{id}. It's actually a typedef to
\texttt{void*}, but is allowable everywhere an object pointer would otherwise
be required.

The syntax for loops, conditionals, and free functions is identical to C. This
allows conditionals to check pointer values directly, without a redundant
\texttt{==0} like Java. Even though free function definitions and calls to
them are identical to C, message sending and method definition is very
different. Those are based on SmallTalk.

\subsubsection{Type System}

As mentioned earlier, all of C's primitives are supported. Structures are also
supported. Wrappers for them exist, although fewer in number than Java.
\texttt{NSNumber} covers all numeric types. \texttt{NSString} and
\texttt{NSMutableString} cover strings. Through them, one can put primitive
types inside containers and use them as regular objects.

Beyond that, Objective--C has two interrelated constructs, the
\texttt{@interface} and the \texttt{@implementation}. The former
declares the members and the list of methods visible to clients. The latter
contains all the method implementations. Note that additional methods can be
defined in the \texttt{@implementation}, which will be added to the method
list as any other. These additional methods can be called by any client due to
the dynamic messaging system.

C's arrays are immediately available. Furthermore, \texttt{NSArray} and
\texttt{NSMutableArray} provide object wrappers around fixed arrays.
Objective--C's containers all provide useful abstractions upon all their
contents, and these two types enable them for arrays. Arrays of arrays are
allowed in C's traditional arrays and in Objective--C. \texttt{NSArray} and
\texttt{NSMutableArray} allow them as well by recursive embedding.

Objective--C's can operate with weak or semi--weak type checking. The original
libraries and runtime use the type \texttt{id} to refer to any object. Any
message could be sent to it, and any type is convertible to \texttt{id}.
Internally, \texttt{id} is a \texttt{typedef} to \texttt{void*}. However,
users complained about the lack of static error checking, so some basic
capabilities were added.

A pointer can be declared to any \texttt{@interface}, and messages sent to
that pointer are checked against the \texttt{@interface}'s declared API (not
the set of messages the \texttt{@implementation} defines). Any messages sent
to the object, but not declared by its API, are flagged as warnings by the
compiler. Because the object could easily respond to many more messages than
it declares, a compilation--stopping error would be inappropriate.

For primitive types, the type checking is exactly the same as standard C.
Traditional \texttt{(type)} casts easily remove any hindrances provided by the
type system.

Objective--C's object model is modeled after SmallTalk. Only single
inheritance is supported, but with two other important features: categories
and protocols.

Categories were originally designed to let developers split up the
\texttt{@implementation}s of types across several source files. They
allow additional methods to be added to an existing type. Furthermore, a
category's implementation of a method will \emph{override} the original. Note
that this isn't the same as overriding a method through inheritance; the
category's methods are part of the original type. The overridden method is
never used. If two or more categories implement the same method, then the
selection of the method implementation that runs depends upon the load order
of the categories' respective object definitions at run--time.

Protocols work essentially like \texttt{interface}s in Java: a set of methods
that are declared but undefined. Classes declare their conformance to one or
more protocols and implement their methods. Due to the fairly untyped nature
of Objective--C, the actual declaration of conformance is mostly for
documentation reasons. Just as often as protocols are used, classes will
declare \emph{informal protocols}. Informal protocols exist only in
documentation as a set of methods a class may choose to implement.
Objective--C's runtime allows a class's implemented method list to be checked,
so a client can see if it implements a method from an informal protocol before
sending it the appropriate message.

At the heart of Objective--C's runtime is the \texttt{Class}. The
\texttt{Class} contains the full description of a class: its superclass, name,
version, size, instance variables, method list, and conformed protocols. The
\texttt{Class} is a \texttt{typedef} to \texttt{struct objc\_class*}, a C
structure. While the definition of \texttt{objc\_class} is available through
the header files, several C and Objective--C functions and methods are
available to the programmer for abstracted access.

The \texttt{Class} contains the list of methods implemented by the type. All
method names are hashed to an opaque \texttt{struct objc\_selector*},
generally just called a \emph{selector}.

Every instance of the same name, no matter which interface or implementation
it exists in, is hashed to the same value. This way, message names have no
binding to any particular class. Using this property, one can send any message
to any type.

Upon compilation, the compiler links to the runtime in two ways. First, the
\texttt{Class} structures are all defined and filled. Second, every message
send has the method name converted to a selector, and then passed to the
runtime function \texttt{objc\_msgSend}.

\texttt{objc\_msgSend} searches the recipient's method list for the selector,
and dereferences the listed pointer to the proper code. The method list is
actually a hash table for speed. If the object doesn't implement the method,
all of its ancestors are searched. If none of them implement it, then it sends
a ``second chance" invocation, by sending \texttt{forwardInvocation:} to the
original recipient.

\texttt{forwardInvocation:} allows an object to handle any message it wants.
The name implies the most common use for this feature: to let an object act as
a proxy for another. An object can implement \texttt{forwardInvocation:} as a
quick check upon its proxied object to see if it responds to the parameter,
and if so, sends the message onwards. \texttt{forwardInvocation:} has a
default implementation in the root class \texttt{NSObject} that throws an
exception and logs the failed message call.

Objective--C doesn''t provide any type of parameter overloading at all. In
fact, parameter checking isn't a strong feature at all in Objective--C. Only
when the optional static type checking features are used can parameter checks
be done at all.

A message is dispatched upon its selector, which is based only on the message
name. When static type checking isn't used, the return type is assumed by the
runtime to be of type \texttt{id}. Any client sending a message to a type that
returns anything else has to cast the value.

As mentioned earlier, a form of overloading occurs with categories that define
methods with the same name as one in the original type. This is a dangerous
practice, as another category could do the same and ambiguate the selection
process.

Objective--C only allows access control upon members. Although one can ``hide"
methods by defining them in the \texttt{@implementation} without declaration
in the \texttt{@interface}, instances will still respond to messages of that
name from any sender.

Members can still be \texttt{@public}, \texttt{@protected}, or
\texttt{@private}. The syntax for accessing them is identical to accessing a
structure member via a pointer. For further control, the traditional C methods
of defining pointers to undefined (such as \texttt{objc\_selector}) or
unlisted types (\texttt{void*}) are still useful.

As mentioned earlier, Objective--C allows a type to inherit from one other.
All the member variables and methods are inherited. \texttt{@private} members
are part of the subclass's definition but are inaccessible. All methods are
inherited and accessible. Any method can be overridden, as there is no concept
of a \texttt{final} or non-virtual method in Objective--C.

While upcasting is directly allowed, it's not very useful. C allows arbitrary
casting and Objective--C doesn't really need a type's name to send it a
message. Except for cases of voluntary static type checking, inheritance in
Objective--C is essentially useful as implementation inheritance only.

Objective--C provides the most flexible polymorphism of any language described
here. It has a binding mechanism much more dynamic than most OO languages.
Furthermore, it allows the \emph{client} much more control over an object's
behavior than most other languages: categories and dynamic message proxies let
the client modify the interface and behavior of the type. Only
\texttt{@private} data members aren't accessible to client code via
subclassing or categorization.

Exceptions only appeared in their current form in Apple's compiler in Mac OS X
10.3. An earlier form existed that was based on macros, but it won't be
described here. The current syntax is very similar to C++: a \texttt{@try}
block encloses code that may \texttt{@throw}, followed by one or more
\texttt{@catch} blocks, each with a typed parameter. Like Java, an optional
\texttt{@finally} block can follow, with code that will always be run.

Only subclasses of \texttt{NSException} may be thrown. The \texttt{@throw}
construct begins a scan and unwind process on the stack for a \texttt{@catch}
block that will handle the exception type or one of its ancestors. Any
\texttt{@finally} blocks are run along the way.

Objective---C doesn''t have a compile--time generics system. Its dynamic
messaging system helps mitigate the loss. For example, the library containers
all support the method \texttt{makeObjectsPerformSelector:}, which sends a
message with the parameter selector to every contained element. Furthermore,
the weak and optional static typing mechanism reduces the value of a generics
facility at all.

\subsubsection{Runtime}

As covered earlier, the runtime plays a large part of the Objective--C system.
At startup, all the loaded classes' \texttt{Class} descriptions are
initialized. From there, the standard C entry point \texttt{main} is run.

Memory is best described as ``semiautomatic." Its inherently
reference--counted, with the appropriate support infrastructure directly
within \texttt{NSObject}. Two methods, \texttt{retain} and \texttt{release},
increase and decrease a reference count in the recipient object. When that
count reaches zero, the object calls \texttt{dealloc} on itself, and releases
any resources it holds. \texttt{dealloc} will then call \texttt{release} on
any member objects it has, causing the proper cascading effect.

For additional functionality, Objective--C supports an
\texttt{aut\-or\-el\-ease} message. \texttt{autorelease} will add the receiver
to the current autorelease pool, an instance of \texttt{NSAutoreleasePool}, a
container. \texttt{NSAutoreleasePool} takes ownership of the object. When the
pool is itself \texttt{release}ed, it will call \texttt{release} on all of its
contents.

\texttt{NSAutoreleasePool}s can be nested within each other, and are
automatically supported by the GUI library in Objective--C, \texttt{AppKit}
(now called Cocoa\cite{cocoa}). The most common use is within the event loop:
an autorelease pool is created when an event enters the application, and
\texttt{release}ed when the event loop reenters it waiting state. In the
meantime, any \texttt{autorelease}d object will have a convenient lifetime
that lasts just long enough to be used for handling the event. Any object that
needs to live longer can be \texttt{retain}ed by any other object, which will
hold the only reference count to it when the event loop reenters the waiting
state.

While not as large as Java''s, Objective--C''s library is certainly quite
powerful. More importantly, it's much more flexible thanks to good leverage of
the runtime's dynamism. For example, the serialization mechanism directly
allows connections between objects to be serialized easily, allowing a GUI
builder to directly connect data objects to GUI components without
necessitating the generation of code or other glue. Furthermore, serialization
works for nontree structures, thanks the the runtime's ability to directly
analyze the objects.

Of particular interest in the library is the single--threaded nature of it
all. The entire system was designed to run within a single thread, with a
messaging bridge to connect threads together. The messaging bridge works
exactly the same as a full interprocess communication system, in fact
identically to its distributed objects mechanism. That will be described
below.

Most of the runtime's abilities come from two places: the messaging system and
functionality built into \texttt{NSObject}. \texttt{NSObject} provides the
following APIs:

\begin{enumerate}

 \item \texttt{conformsToProtocol:} --- If the object exports a specific
protocol's API.

 \item \texttt{respondsToSelector:} --- If the object responds to a specific
method.

 \item \texttt{isKindOfClasss:} --- If the object has a specific class as its
type or in its type inheritance hierarchy.

 \item \texttt{isMemberOfClass:} --- If the object is of a specific type.

\end{enumerate}

These methods constitute around 90\% of what's needed by most applications.  The remainder go and directly analyze the \texttt{objc\_class} structure for more information.  For the most part, the dynamic messaging infrastructure and weak typing reduce the need for direct introspection in the system.

\subsubsection{Case Study: Distributed Objects}

There need not be any ``what if" scenarios here; Objective--C's basic runtime
libraries provide distributed object (DO) functionality. In fact, this
functionality's used for more than interprocess communication; Objective--C
uses its DO mechanism for inter-thread communications; avoiding the
traditional complexities of synchronization.

The distributed object system works quite simply: a generic proxy class for
the transport mechanism masquerades as the type it's proxying. The proxy tells
its clients that it responds to all the selectors of its destination. Any
messages sent to it are serialized and sent through the transport to the
recipient. The system is so simple because of both the flexible messaging
system and the built--in serialization system.

The serialization of messages is significant; such a serialized form could be
replicated, replayed, or even analyzed and restructured. For a DO system,
another question arises, when should a parameter be serialized and when should
it be proxied? If the parameter is a simple, flat data structure, it can and
should be serialized. However, if it's a complex object with relationships to
other objects, it should be proxied. Serializing and restoring such an object
will certainly be complex, possibly lossy. Such a determination also can't be
made by the compiler; these are design--level decisions.

For that reason, additional keywords exist in Objective--C to tag parameters
that should be proxied. The terms \texttt{byref} and \texttt{bycopy} indicate
which parameters are sent by reference and copy, respectively

\subsubsection{Case Study: Serialization}

Serialization is built into Objective--C as well; it's a prerequisite for
marshaling a message. There are two parts to serialization: wrappers for each
primitive type and a protocol for serialization. The latter, \texttt{NSCoding}
lets serializing objects interoperate and cooperate for more complex types.
The wrappers implement the protocol and take most of the grunt work out of
serialization.

As Objective--C doesn't allow the iteration of an object's members, there is
no support for automating the serialization process of a complex type; some
code has to be written. In practice, the process is often much simpler than it
sounds. Objective--C's runtime library is full of high--quality containers
that will serialize themselves and all their contents. Often, it's
advantageous to use one or two containers instead of many members; letting
them do most of the serialization work.

%
%
%

\subsection{Java}
%-   Intro                       
%  -   Influence                     
%  -   History                     
%  -   Intent                      

Java\cite{java} is a programming language derived from Objective--C and C++.
Developed at Sun Microsystems, it has become extremely popular for World Wide
Web applications. Originally intended as an embedded systems language, one of
Java's original goals was platform independence. That goal has lead to a
toolchain and runtime that runs on the vast majority of the worlds' desktop
computers and some of the larger embedded platforms. Linux, Windows, MacOS,
Symbian, PalmOS, and many variants of Unix are just a few of the operating
systems Java runs on today.

Due to its ease of access, strong APIs, and simple programming model, Java has
become one of the dominant languages used today. Java's virtual machine (VM)
executes Java Byte Code, an assembly--like language atop of a stack--based
virtual processor. However, the virtual machine has some unusual intelligence
to it: specifically that it knows about objects. Java is thoroughly
object--oriented, and that shows in the virtual machine's architecture.

The virtual machine starts and loads up classes from a \texttt{ClassLoader},
an object responsible for loading class definitions into the system for use.
\texttt{ClassLoader}s can be custom--written and added to the system to work
in unison with the default file--based loader, allowing networked access or
on--the--fly class generation\cite{javassist}.

Beyond knowing how to load segments of code in classes, Java's virtual machine
has another capability granted from its knowledge of the object model: garbage
collection. The Java language is fully and transparently garbage collected.
Unreferenced objects, even those who keep cycles of references to each other,
are properly destroyed and their memory reclaimed during the normal execution
of a Java program.

Modern virtual machines are used in high--load, mission--critical
environments. They have become very solid and their performance has been
extensively optimized. Today's VMs watch for often--used sequences of Java
bytecode and compile them into native machine instructions for further
optimization. Also, completely--native compilers exist for Java\cite{GCJ} that
eliminate the need for a virtual machine and its overhead.

\subsubsection{Compilation}

Before a class can be loaded, it has to be compiled. Compilation occurs before
program execution. Each class in Java has its own \texttt{.java} file that is
compiled into a \texttt{.class} file. Each \texttt{.java} file can only have
one publicly--accessible class in it, defined with the same name as the
filename. All static checks are performed, inside a class and between classes.
Because classes are compiled into separately--loaded files, additional checks
are needed at runtime to make sure the file is compatible with the system.

In general, run--time compatibility errors show up as accesses to undefined or
inaccessible members or methods of a class. These failed access attempts exist
as exceptions, which can be caught. Due to the need to catch these kinds of
errors, and other reasons, Java saves a rich amount of information about each
compiled class directly within the \texttt{.class} file, which is loaded and
maintained by the virtual machine during execution. Java's introspection
mechanism exposes this information to the programmer.

Even though there isn't a true linking stage within the Java compilation
model, there's still a need to distribute single files for entire applications
or libraries. The Java utility \texttt{jar} acts almost exactly like the
standard Unix \texttt{tar} utility; only that it uses the \texttt{.zip} file
format and stores a manifest file inside the archive. The Java virtual
machine's standard \texttt{ClassLoader} can directly read \texttt{jar}s and
load class code from them.

%  \texttt{jar} files can contain more than just executable code: text files, resources, graphics, and anything else can be saved in them.  The \texttt{ClassLoader} will ignore them, but they can be accessed by the code within the \texttt{jar}.  Such an ability enables the single--file distribution of an entire Java application.  Java's J2EE (Java 2 Enterprise Edition) uses such a distribution model for its web applications, which deploy as a single file that's loaded by the application server and executed.  Java's WebStart application allows a simple URL to specify a remote application, which is loaded on demand and cached locally.  WebStart removes the need for an application to have a traditional installation phase, and always keeps the cached version up--to--date; greatly simplifying desktop maintenance across an organization.

%-   Compilation                   
%  -   Compilation process               
%  -   Preprocessing stages                
%  -   input & output                    
%  -   archives & libraries                  
%  -   application management           

\subsubsection{Syntax}

Java's syntax for basic procedural use is very similar to C and C++. However,
it cannot be used without using the object--oriented features. As implied by
the compilation process, Java only compiles classes. All code must exist
within classes, either in methods, constructors, or static initialization
blocks.

To demonstrate, take a look at Figure \ref{fig:java-simple}. Like C, the entry
point for any program in Java is \texttt{main()}. However, Java doesn't allow
any code to exist outside of a class definition, hence the need for the
otherwise--useless \texttt{MainContainer}.

The syntax for declarations, loops, and conditionals is nearly identical to C
and C++. The standard set of \texttt{for}, \texttt{while}, and \texttt{do}
loops exists. Two main differences exist: (1) statements given as parameters
to conditionals must be of \texttt{boolean} type, and (2) there are no
explicit references or pointers. For example, the comparison \texttt{(i==0)}
could be written as simply \texttt{(i)} in C or C++; if all the bits of
\texttt{i} are zero, then \texttt{i} is considered \texttt{false}. Also, the
variable \texttt{mc} is a reference to a heap--allocated
\texttt{MainContainer}. Unlike C++, no \texttt{*} or \texttt{\&} is necessary.

For primitive data types, the declarations are nearly identical to C and C++.
Every other declaration is implicitly a reference to an object. References to
references are not allowed. Although the references keep the same lifetime as
primitive types, the lifetimes of the objects are different. All objects live
on the heap through a call to \texttt{new}, and die upon garbage collection.

\begin{figure}[ht!]
\begin{verbatim}
public class MainContainer {
  int loop = 32;
  
  public static void main (String args[]) {
    MainContainer mc = new MainContainer ();
  for (int i=0; i<loop; i++) {
    System.out.println ("Hello World");
    if (i == 0) {
      System.out.print ("!-");
    }
  }
  }
}
\end{verbatim}
\caption{Simple Java Program}
\label{fig:java-simple}
\end{figure}

%-   Syntax                      
%  -   Overview                      
%  -   Procedural syntax                 
%  -   Declarations                    
%  -   Conditionals                    
%  -   Loops                     
%  -   Procedure calls                 

\subsubsection{Type System}

Java has taken a pick--and--choose approach to its type system from C++ and
Objective--C. The primitive types come from the latter. Primitive types like
\texttt{int}, \texttt{float}, \texttt{double}, \texttt{boolean}, and
\texttt{char} all exist in Java and behave as they do in C, C++, or
Objective--C. The only difference is that their specific sizes and precisions
are completely defined; unlike Objective--C or C++ where it varies upon the
specific processor architecture.

Similar to Objective--C's \texttt{NSNumber}, Java has fully--fledged class
types that peer the primitives. \texttt{Integer}, \texttt{Float},
\texttt{Double}, \texttt{Boolean}, and \texttt{String} all provide wrappers
around a primitive value, as well as comparison and conversion operations. As
discussed later, Java's containers can only contain full objects, and these
peers allow the primitive types to be used.

Java doesn't have the concept of an explicit pointer or reference type;
instead, every variable declared to be of a class's type is really a reference
to an instance allocated on the heap. Every such variable has to be
initialized with a call to \texttt{new}, and any other variable set to be of
the same value will refer to the same object. While references to objects can
be declared \texttt{final} like primitive types, making them immutable, the
referred objects are always mutable. To simulate a const object as in C++, the
traditional Java method to return immutable objects is to define a subset of
the object's interface with only ``getter" methods that allow the query but
not modification of an object's state. The true object's exposed interface is
a superset of this, and it formally ``implements" it (discussed more later).

Arrays in Java are essentially unidimensional. They hold either primitive
types or references to objects. However, arrays are also objects, and thus
references to arrays can be stored in arrays as well. Through this
double-indirect mechanism, multidimensional arrays are implemented in Java.
The immediate benefit of the flexibility is clear: arrays are simple to
understand and flexible to use. For example, the references to objects could
be polymorphic; allowing further dynamism.

Arrays, like other Java objects, are always mutable and garbage collected.
They also ``know" their length, and can be queried for them as needed. Arrays
also have a peer, \texttt{Array}, which provides similar wrapping facilities
as the other peers.

Java uses a similar type system as C++. All variables are declared to have
some type, either primitive or object. Attempts to assign the variables values
of other, incompatible types are errors. Downcasting from a class to one of
its subclass types is allowed. However, there is little allowance for implicit
conversion; only upcasting. Even narrowing conversions between floating point
types from literals are errors!

\subsubsection{Objects}

Java's object model is a mix between C++ and Objective--C. The methods are
declared and used almost identically to C++ syntax. However, the inheritance
and polymorphic mechanisms within Java function more similarly to
Objective--C.

Messages in Java look and act almost identically to C++: essentially functions
with a hidden \texttt{this} pointer back to the object. An attempt to call a
method not implemented by the message recipient is flagged as a compile--time
error. Furthermore, methods can be overloaded: more than one method can be
declared and defined with the same name, as long as their parameter list
differs.

Unlike C++, default values for parameters, nor the overloading of operators
are allowed. There is one exception: Java's \texttt{String} class has
concatenation operators defined, with definition for all the primitive types
and behavior to call \texttt{toString} on all other objects given as
parameters.

Class types fall into three categories: classes, abstract classes, and
interfaces. Classes have all methods defined, member variables, and are
constructible. They can also inherit from one other class, and
\emph{implement} any number of interfaces. Interfaces only have methods
declared without implementation. Classes implement interfaces by implementing
all of their methods.

Abstract classes are unconstructable objects with one or more methods
unimplemented and marked \texttt{abstract}. Unlike interfaces, they can have
some methods defined for subclass use, but take the role as the only
superclass. Subclasses must implement all abstract methods of their abstract
superclass and any implemented interfaces to be constructable.

For classes that have non-memory resources allocated, a \emph{finalizer} can
be defined, which is run when the object is garbage collected. However, the
specific time of execution, or even a guarantee of execution, isn't provided.

Access control is extremely C++-like: private, public, and package access is
allowed. Access sections like C++ aren't provided; each member and method has
to have its own qualifier listed, otherwise it's assumed to be package--level
access. Inheritance and interface implementation, however, are always public.

Objects can be queried of their type through several ways. First, the
\texttt{instanceof} operator returns a boolean value specifying if the object
is an instance of a specific type or subtype thereof. Next, \texttt{Class}
provides the comparison methods \texttt{isInstance} and
\texttt{isAssignableFrom}, both of which compare compatibility with another
object. Most often, the \texttt{instanceof} operator is used.

As mentioned before, classes can inherit from exactly one superclass. When one
isn't mentioned, it's Java's standard \texttt{Object}. As a consequence, every
object in the Java system inherits from Object. Without multiple inheritance,
questions about diamond inheritance, ambiguous superclass references, and the
like are completely avoided. Also as mentioned before, each compiled class has
a description that's loaded, maintained, and checked by the virtual machine.
Such a description is available to the developer as an instance of type
\texttt{Class}, available through \texttt{Object}'s \texttt{getClass} method.

All methods are implicitly polymorphic; the C++ keyword \texttt{virtual} is
assumed. The Java keyword \texttt{final} will specify a method that cannot be
overridden in base classes. For a method to be overridden, it has to have the
same access level as the original, and the same signature (method name and
parameter types). As upcasting is an implicit conversion, Java objects are
often treated as if they were instances of their base class or an implemented
interface. Method calls to are routed to the closest ancestor's
implementation.

Exceptions are based on C++: a \texttt{try} block containing code that may
throw, one or more \texttt{catch} blocks that handle a specific type of
exception, and specific to Java, a \texttt{finally} block for cleanup code
that runs even if a stack unwind is in progress.

Unlike C++, Java's exceptions only use class types: throwing an integer or
floating point value isn't possible. The virtual machine is a source of many
exceptions as well. Illegal actions, such as trying to call a method on a null
reference or casting an object not of the specified type, are trapped by the
VM and result in exceptions being thrown in the running program. This gives
the running program a reasonable chance to trap the error and continue
execution.

Java has a generics system in its 5.0 beta as of August
2004\cite{java-generics}. Java's compiler has a simple preprocessor that
allows the declaration of generic types. Generic types in Java have a similar
syntax as C++: angle brackets denote parameters to the generic type. The
parameters are used to denote types used for method parameters, member types,
and return types. The compiler will flag attempts to use an instantiation of a
generic type that doesn't match its definition. For example, a generic
container will not allow insertions of objects that aren't instances or
subclasses of its parameter.

Behind the scenes, the generic types have only one instantiation that's shared
between all uses. The parameter type names are all converted to
\texttt{Object}, and compiled as a normal Java class. This way, the
traditional one--to--one mapping of a \texttt{.java} source file and the
compiled \texttt{.class} still exists. Unfortunately, the only gains from the
generics feature are some type safety and reduced need for casting; none of
the more powerful capabilities generics provide in C++ are available in Java.

%-   Type System                   
%  -   Primitive Type Handling (integer, Integer, etc)       
%  -   Arrays                      
%  -   Type Safety/Checking                
%  -   Object Model                    
	%  -   Messaging                   
	%  -   Overloading                   
		%  -   Operators                     
	%  -   Access Control                    
	%  -   Inheritance                   
	%  -   Polymorphism                    
%  -   Exceptions                    
%  -   Generics                      

\subsubsection{Runtime}

As mentioned before, the virtual machine loads Java \texttt{.class} files via
a \texttt{ClassLoader}, which returns a \texttt{Class} object to the VM. At
startup, the VM is given a single class name to load, which must have a
\texttt{public}, \texttt{static} method named \texttt{main} taking a single
parameter: an array of \texttt{String}. That method is run with the command
line options given at the VM's invocation, with VM--specific options removed.

\texttt{main} may spawn any number of threads, which are supported natively by
the VM. Included with the ability to create new threads are in--language
synchronization abilities, such as the ability to make a method
\texttt{synchronized}: callable only from one thread at a time. The virtual
machine provides the threading and enforces the synchronization, even if the
underlying platform doesn't do it natively.

Like C, C++, and Objective--C, the program lives only as long as \texttt{main}
runs, even if other threads are still active when \texttt{main} completes.
During that lifetime, all memory allocated is tracked and managed by the VM.
Objects and graphs thereof with no incoming references are garbage collected
and their memory reclaimed.

One of Java's most powerful assets is the wealth of standard libraries. The
standard library contains nearly 2,000 classes and interfaces in nearly 100
packages. Together, a platform--independent system for developing desktop,
web, and command--line applications exists, with facilities for almost every
common development need.

Due to the easy packaging and distribution of platform--independent code, Java
also has one of the richest 3rd party library communities.

Within the standard libraries lie Java's introspection mechanism. The
mechanism's libraries allow programmatic access to the class information the
VM maintains. As hinted before, the \texttt{Class} type provides the key
interface for accessing a type's information. With it, members, methods,
interface, and superclass information is all available. Moreover,
\texttt{Class} provides a query function for getting the appropriate
\texttt{Class} instance for a type with a specific name. Such an ability,
connected with the \texttt{ClassLoader} mechanism, allows a Java program to
assimilate code that was not available at the original system's compilation,
loading it, linking it, and running it when appropriate.

\texttt{Class} provides a healthy API, we present only the relevant subset for
this discussion. All the \texttt{public} constructors, methods, and members
are accessible via \texttt{getConstructor}, \texttt{getMethod} and
\texttt{getField}. Plural versions of these methods exist that return arrays
of each as well. These methods return \texttt{Constructor}, \texttt{Method},
and \texttt{Field} objects.

\texttt{Constructor} is essentially a factory class. Given the parameters it
needs for initialization, its \texttt{newInstance} will return a newly
constructed instance of the type. It allows querying of the required parameter
types via \texttt{get\-P\-ar\-am\-et\-er-\-T\-yp\-es}, which returns an array
of \texttt{Class} objects. Note that primitive types do have \texttt{Class}
objects, but they must be passed to \texttt{Constructor} as wrapped objects.

\texttt{Method} acts almost as a selector in Objective--C. Similar to
\texttt{Constructor}, it has a \texttt{get\-Par\-am\-et\-er\-T\-yp\-es} method
for accessing the parameters. It also has \texttt{getReturnType} and
\texttt{getName} for getting the full method description. \texttt{invoke} in
\texttt{Method} takes a recipient object and a set of parameters and invokes
the method on the recipient.

\texttt{Field} provides \texttt{get} and \texttt{set} methods for objects and
pairs of these methods for each primitive type. All of them take a recipient
object which contains the member variable in question. The primitive type
pairs are named as \texttt{getInt} and \texttt{setInt} which return and take
primitive types for primitively--typed members.

All three classes, \texttt{Constructor}, \texttt{Method}, and \texttt{Field}
will throw exceptions when they are used inappropriately. Examples include
passing the wrong parameter types, sending to the wrong object, or passing
invalid values. Because the classes must have type--agnostic interfaces, these
errors cannot be caught by the compiler.

%-   Runtime                     
%  -   Startup                     
%  -   Memory Management               
%  -   Libraries                     
%  -   Introspection                   

\subsubsection{Case Study: Serialization}

Java provides built--in serialization\cite{957325,944589,376846}. By
implementing a zero--method interface \texttt{Serializable}, an object can be
serialized. The primitive types can also be serialized.

Use \texttt{Obj\-ect\-Outp\-utStr\-eam}\cite{java-api}, a wrapper around a
normal Java \texttt{Out\-p\-ut\-Str\-eam}, to serialize the object.
\texttt{Obj\-ect\-Outp\-utStr\-eam}'s \texttt{writeObject} method will
serialize the object to the stream, and its peer
\texttt{Obj\-ect\-Inp\-utStr\-eam}'s \texttt{readObject} will deserialize it.

By sitting atop of the standard stream mechanism in Java, serialization works
atop of any byte--stream I/O mechanism. Included in the standard libraries are
files and sockets. The developer may write their own stream classes and send
serialized objects over them with little difficulty.

Serialization is almost completely transparent to the object being marshaled.
The only time a class need worry about serialization is when it's got members
that don't implement \texttt{Serializable}. While most of Java's classes do,
some don't for obvious reasons, like \texttt{Thread}. For these members, the
class must mark them \texttt{transient} in their declaration.

Furthermore, the class may need to know when it's being serialized or
deserialized, so that it can adjust its state. For example, it would have to
reconstruct any \texttt{transient} members that were lost during
serialization. The class can define private \texttt{readObject} and/or
\texttt{writeObject} methods which will be called during the relevant
processes. From there, it can prepare for serialization or fully restore from
it.

Another option in Java is the \texttt{Externalizable} mechanism, which does
less of the work by itself, in exchange for greater control of the
serialization format.


\subsubsection{Case Study: Distributed Objects}

Java provides a basic distributed objects mechanism called Remote Method
Invocation (RMI)\cite{java-rmi}. By implementing a zero--method interface
\texttt{Remote}, an object specifies that it can be remotely invoked.

When a message is sent to a remote object, the parameters are serialized.
Those parameters which implement \texttt{Remote} are given
remotely--accessible identifiers, which are sent in their place. A parameter
that implements neither \texttt{Remote} nor \texttt{Serializable} can't be
sent. From there, the virtual machines interact to transport and dispatch the
message.

\subsection{OpenC++}
	% Introduction
	OpenC++\cite{chiba, opencpp, opencpp-tut} is an extension upon C++ that allows metaclasses to be defined, that plug into the compiler.  It's been used as the basis for a query--based debugger\cite{263752, 381750}.  These metaclasses define a translation layer between the input source code (in an extended C++ called simply OpenC++) and the C++ language proper.  The metaclass system allows full awareness of the structure of defined types (introspection) and of the context of each use of the defined types.  The metaclasses can modify both as the source code is compiled.

	OpenC++ is a general--use system.  A similar approach\cite{240738} has been used specifically for the high--performance computing arena.

	% Compilation Model
	\subsubsection{Compilation}
	The compilation system is dynamic: the input source code is compiled into plugins for the compiler, which then monitor and modify the compilation of itself.  Currently, only a single stage is allowed: the input source code is OpenC++, and the metaclasses must emit standard C++.  An extension is planned to allow metaclasses to emit OpenC++, which is then fed into additional passes through the metaclasses for them to emit C++.

	The metaclasses have two parts.  First a \texttt{TypeInfo} object is generated for every declared type, describing its methods, members, and inheritance hierarchy.  Second, a metaclass is its own OpenC++ type derived from \texttt{Class}, containing event handlers that plug into the compiler.  The handlers are called when the compiler encounters the type itself or any uses of it.  The handlers are fed \texttt{PTree}s: constructs similar to Lisp S-expressions that describe the parse tree of the code being compiled.  The handler can modify the \texttt{PTree} before returning it to the compiler.  The handlers use the \texttt{TypeInfo} as needed to determine the necessary changes to the \texttt{PTree}.

	% Introspective Abilities
	\subsubsection{Introspective Abilities}
	\texttt{TypeInfo} stores the traditional C++ structural definition of a type: its members, methods, and inheritance hierarchy.  As such, it provides all the information we need for an introspective mechanism.  However, it's only available for use to metaclasses in OpenC++ and runtime code in C++; there's no integration or availability to the template mechanisms within C++.
	
	To connect the introspective data to the template mechanism in C++; one has to write metaclasses that generate type traits describing the information found in the respective \texttt{TypeInfo}.

	% Uses
	\subsubsection{Case Study: Distributed Objects}
	OpenC++ allows a simple implementation for the basic network distribution of objects.  A metaclass can simply handle calls to the type's methods, inserting any desired marshaling and I/O code necessary in the call's place.  Alternatively, a metaclass could generate a full stub class to act as a local proxy for the remote object.  

	In OpenC++, a programmer can define new keywords that automatically specify the metaclass.  For example, the programmer could register a keyword \texttt{distribute} to specify a \texttt{DistributedObject} metaclass.  A simple ``hello world" class is shown in Figure \ref{fig:ocpp-dist}.

\begin{figure}[ht!]
\begin{verbatim}
distribute class Greeter {
public:
    std::string getMessage ();
};
\end{verbatim}
\caption{OpenC++ Class with Distribution Keywords}
\label{fig:ocpp-dist}
\end{figure}

	\texttt{Greeter}'s metaclass would be \texttt{Dis\-tr\-ib\-ut\-ed\-Ob\-j\-ect}. \texttt{Dis\-tr\-ib\-ut\-ed\-Ob\-j\-ect} would make \texttt{get\-Mes\-s\-age} \texttt{virtual}, create a subclass called \texttt{Greeter\-\_proxy\-\_ala\-\_Dis\-tr\-ib\-ut\-ed\-Ob\-j\-ect}, and generate any glue necessary for the runtime distributed messaging system to let it create and configure the proxy.

	\subsubsection{Case Study: Serialization}
	OpenC++ makes easy work of serializing a type.  The metaclass can add new methods to the type to serialize and deserialize it.  These methods can serialize as much as they can, pushing off any other parts to run--time code.  

	A type will typically have several categories of members to serialize:
	\begin{enumerate}
	\item \emph{Simple Members} --- Primitive types that can be directly serialized.  Such as integers, floats, and C strings.
	\item \emph{Irrelevant Members} --- Members that need not be serialized.  For example, caches.
	\item \emph{Simple References} --- Pointers to other objects completely owned by its single container, but in need of run--time support to serialize.  For example, a dynamically--allocated array.
	\item \emph{Complex References} --- Pointers to objects that may already have been serialized, or may transitively refer back to the container.  For example, a graph of objects with cycles in it.  In such cases, a placeholder token may be needed to refer to an object already serialized.
	\end{enumerate}

	For each type, a keyword can be added.  For example, the class in Figure \ref{fig:ocpp-class-keyw} has members of each type.

\begin{figure}[ht!]
\begin{verbatim}
persistent class DataObject {
    // simple members
    int a, b;
    
    // irrelevant members
    transient int c,d;
    
    // simple references
    single (serialize_array) int string_length;
    single (serialize_array) char * string_body;
    
    // complex references
    shared DataObject *parent;
    
    size_t serialize_array (mode_t inorout,
                            memory_buffer &buffer,
                            int &val_string_length, 
                            char *&val_string_body);
};
\end{verbatim}
\caption{OpenC++ Class with Serialization Keywords}
\label{fig:ocpp-class-keyw}
\end{figure}

	The metaclass would have to keep track of which \texttt{DataObject}s were already serialized, and simply refer to them as needed.  For the simple references, the metaclass's serialization methods would call \texttt{serialize\_array} to bring the data values in and out of the memory buffer to the parameter member references (\texttt{val\_string\_length} and \texttt{val\_string\_body}).



